% This is an example of using latex for a paper/report of specified
% size/layout. It's useful if you want to provide a PDF that looks
% like it was made in a normal word processor.

% While writing, don't stop for errors
\nonstopmode

% Use the article doc class, with an 11 pt basic font size
\documentclass[11pt, a4paper]{article}

% Makes the main font Nimbus Roman, a Times New Roman lookalike:
%\usepackage{mathptmx}% http://ctan.org/pkg/mathptmx
% OR use this for proper Times New Roman (from msttcorefonts package
% on Ubuntu). Use xelatex instead of pdflatex to compile:
\usepackage{fontspec}
\usepackage{xltxtra}
\usepackage{xunicode}
\defaultfontfeatures{Scale=MatchLowercase,Mapping=tex-text}
\setmainfont{Times New Roman}

% Set margins
\usepackage[margin=2.5cm]{geometry}

% Multilingual support
\usepackage[english]{babel}

% Nice mathematics
\usepackage{amsmath}

% Control over maketitle
\usepackage{titling}

% Section styling
\usepackage{titlesec}

% Ability to use colour in text
\usepackage[usenames]{color}

% For the \degree symbol
\usepackage{gensymb}

% Allow includegraphics and nice wrapped figures
\usepackage{graphicx}
\usepackage{wrapfig}
\usepackage[outercaption]{sidecap}

% Set formats using titlesec
\titleformat*{\section}{\bfseries\rmfamily}
\titleformat*{\subsection}{\bfseries\itshape\rmfamily}

% thetitle is the number of the section. This sets the distance from
% the number to the section text.
\titlelabel{\thetitle.\hskip0.3em\relax}

% Set title spacing with titlesec, too.  The first {1.0ex plus .2ex
% minus .7ex} sets the spacing above the section title. The second
% {-1.0ex plus 0.2ex} sets the spacing the section title to the
% paragraph.
\titlespacing{\section}{0pc}{1.0ex plus .2ex minus .7ex}{-1.1ex plus 0.2ex}

%% Trick to define a language alias and permit language = {en} in the .bib file.
% From: http://tex.stackexchange.com/questions/199254/babel-define-language-synonym
\usepackage{letltxmacro}
\LetLtxMacro{\ORIGselectlanguage}{\selectlanguage}
\makeatletter
\DeclareRobustCommand{\selectlanguage}[1]{%
  \@ifundefined{alias@\string#1}
    {\ORIGselectlanguage{#1}}
    {\begingroup\edef\x{\endgroup
       \noexpand\ORIGselectlanguage{\@nameuse{alias@#1}}}\x}%
}
\newcommand{\definelanguagealias}[2]{%
  \@namedef{alias@#1}{#2}%
}
\makeatother
\definelanguagealias{en}{english}
\definelanguagealias{eng}{english}
%% End language alias trick

%% Any aliases here
\newcommand{\mb}[1]{\mathbf{#1}} % this won't work?
% Emphasis and bold.
\newcommand{\e}{\emph}
\newcommand{\mycite}[1]{\cite{#1}}
%% END aliases

% Custom font defs
% fontsize is \fontsize{fontsize}{linespacesize}
\def\authorListFont{\fontsize{11}{11} }
\def\corrAuthorFont{\fontsize{10}{10} }
\def\affiliationListFont{\fontsize{11}{11}\itshape }
\def\titleFont{\fontsize{14}{11} \bfseries }
\def\textFont{\fontsize{11}{11} }
\def\sectionHdrFont{\fontsize{11}{11}\bfseries}
\def\bibFont{\fontsize{10}{10} }
\def\captionFont{\fontsize{10}{10} }

% Caption font size to be small.
\usepackage[font=small,labelfont=bf]{caption}

% Make a dot for the dot product, call it vcdot for 'vector calculus
% dot'. Bigger than \cdot, smaller than \bullet.
\makeatletter
\newcommand*\vcdot{\mathpalette\vcdot@{.35}}
\newcommand*\vcdot@[2]{\mathbin{\vcenter{\hbox{\scalebox{#2}{$\m@th#1\bullet$}}}}}
\makeatother

\def\firstAuthorLast{James}

% Affiliations
\def\Address{\\
\affiliationListFont Adaptive Behaviour Research Group, Department of Psychology,
  The University of Sheffield, Sheffield, UK \\
}

% The Corresponding Author should be marked with an asterisk. Provide
% the exact contact address (this time including street name and city
% zip code) and email of the corresponding author
\def\corrAuthor{Seb James}
\def\corrAddress{Department of Psychology, The University of Sheffield,
  Western Bank, Sheffield, S10 2TP, UK}
\def\corrEmail{seb.james@sheffield.ac.uk}

% Figure out the font for the author list..
\def\Authors{\authorListFont Sebastian James\\[1 ex]  \Address \\
  \corrAuthorFont $^{*}$ Correspondence: \corrEmail}

% No page numbering please
\pagenumbering{gobble}

% A trick to get the bibliography to show up with 1. 2. etc in place
% of [1], [2] etc.:
\makeatletter
\renewcommand\@biblabel[1]{#1.}
\makeatother

% reduce separation between bibliography items if not using natbib:
\let\OLDthebibliography\thebibliography
\renewcommand\thebibliography[1]{
  \OLDthebibliography{#1}
  \setlength{\parskip}{0pt}
  \setlength{\itemsep}{0pt plus 0.3ex}
}

% Set correct font for bibliography (doesn't work yet)
%\renewcommand*{\bibfont}{\bibFont}

% No paragraph indenting to match the VPH format
\setlength{\parindent}{0pt}

% Skip a line after paragraphs
\setlength{\parskip}{0.5\baselineskip}
\onecolumn

% titling definitions
\pretitle{\begin{center}\titleFont}
\posttitle{\par\end{center}\vskip 0em}
\preauthor{ % Fonts are set within \Authors
        \vspace{-1.1cm} % Bring authors up towards title
        \begin{center}
        \begin{tabular}[t]{c}
}
\postauthor{\end{tabular}\par\end{center}}

% Define title, empty date and authors
\title {
  Karbowski model in 2-D
}
\date{} % No date please
\author{\Authors}

%% END OF PREAMBLE

\begin{document}

\setlength{\droptitle}{-1.8cm} % move the title up a suitable amount
\maketitle

\vspace{-1.8cm} % HACK bring the introduction up towards the title. It
                % would be better to do this with titling in \maketitle

\section{Introduction}

This is a working document starting from the model described in Eqs.~2
and 4 in ``Model of the Early Development of Thalamo-Cortical
Connections and Area Patterning via Signaling Molecules'',
Karbowski \& Ermentrout, 2004~\cite{karbowski_model_2004}. I then
consider extending this model into two dimensions.

\section{One dimensional system}

Equations 1-4 of \cite{karbowski_model_2004} is the full set of
differential equations describing their system:

\begin{equation}
\frac{\partial c_i(x, t)}{\partial t} = -\alpha_i c_i(x, t) + \beta_i
n(x, t)[a_i(x, t)]^k
\end{equation}
%
\begin{equation}
\frac{\partial a_i(x, t)}{\partial t} = \frac{\partial J_i(x, t)}{\partial
x} + \alpha_i c_i(x, t) - \beta_i n(x, t)[a_i(x, t)]^k
\end{equation}
%
\begin{equation}
n(x, t) + \sum_{i=1}^{N} c_i(x, t) = 1
\end{equation}
%
with the flux current
%
\begin{equation}
J_i(x, t) = D \frac{\partial a_i(x, t)}{\partial x} - a_i
\bigg(\gamma_{Ai} \frac{\partial \rho_A(x)}{\partial x} +\gamma_{Bi} \frac{\partial \rho_B(x)}{\partial x} + \gamma_{Ci} \frac{\partial \rho_C(x)}{\partial x} \bigg)
\end{equation}

Note that (unlike in the original paper) I have been explicit about
the space and time dependence of the state variables $a_i$, $c_i$,
$J_i$, $n$, $\rho_A$, $\rho_B$ and $\rho_C$. Note also that
$i$ \emph{always} refers to the thalamo-cortial connection type; there
is one equation system for each connection type. Keep this in mind
when reading the sums in the maths below. Note finally that I've not
defined all the terms in this working document, but I've used the same
variable names as in the paper~\cite{karbowski_model_2004}.

\section{Two dimensional system}

Extending the system above to two spatial dimensions changes $x$ into
a two-dimensional $\mb{x}$ and changes the flux term, $J_i(x)$ as follows:
%
\begin{equation} \label{eq:Karb2D_dc}
\frac{\partial c_i(\mb{x},t)}{\partial t} = -\alpha_i c_i((\mb{x},t) + \beta_i n(\mb{x},t)
[a_i(\mb{x},t)]^k
\end{equation}
%
\begin{equation} \label{eq:Karb2D_da}
\frac{\partial a_i(\mb{x},t)}{\partial t} = \nabla\vcdot\mb{J}_i(\mb{x},t) + \alpha_i c_i(\mb{x},t) - \beta_i n(\mb{x},t)
[a_i(\mb{x}, t)]^k
\end{equation}
%
\begin{equation} \label{eq:Karb2D_conserve}
n(\mb{x},t) + \sum_{i=1}^{N} c_i(\mb{x}, t) = 1
\end{equation}
%
with the flux current
%
\begin{equation} \label{eq:Karb2D_J}
\mb{J}_i(\mb{x},t) = D \nabla a_i(\mb{x},t) - a_i
\big(\gamma_{Ai} \nabla\rho_A(\mb{x}) +\gamma_{Bi} \nabla\rho_B(\mb{x}) + \gamma_{Ci} \nabla\rho_C(\mb{x}) \big)
\end{equation}

The boundary condition that I'd like to apply to these equations is:
%
\begin{equation}
\mb{J}_i(\mb{x},t) \bigg\rvert_{boundary} = 0
\end{equation}
%
to agree with the boundary condition applied in the paper for the 1D system.

\subsection{Computing div J: approach 1}

In order to compute $\nabla\vcdot\mb{J}_i(\mb{x},t)$, I need to do
a little work. First, I point out that
%
\begin{equation}
\mb{g}(\mb{x}) \equiv \big(\gamma_{Ai} \nabla\rho_A(\mb{x}) +\gamma_{Bi} \nabla\rho_B(\mb{x})
+ \gamma_{Ci} \nabla\rho_C(\mb{x}) \big)
\end{equation}
%
is a static vector field; once computed at the start of the
simulation, it is time-invariant. $\mb{J}_i$ is thus:
%
\begin{equation}\label{eq:J}
\mb{J}_i(\mb{x},t) = D \nabla a_i(\mb{x},t) - a_i \mb{g}(\mb{x})
\end{equation}

Taking the divergence of Eq.~\ref{eq:J}:
%
\begin{equation} \label{eq:divJ}
\nabla\vcdot\mb{J}_i(\mb{x},t) = \nabla\vcdot\bigg(D \nabla
a_i(\mb{x},t) - a_i \mb{g}(\mb{x}) \bigg)
\end{equation}

Now simply multiply the current values of $a_i$ and $\mb{g}$ and
compute an approximation to $\nabla a_i$, resulting in a vector field
which is $\mb{J}$. The boundary condition can be forced by setting
this to 0 for boundary hexes (ugly, and quite possibly completely
fallacious). Finally, compute $\nabla\vcdot\mb{J}_i(\mb{x},t)$ using
the result that divergences can be simplified by means of Gauss's
Theorem. This states that (in a two-dimensional system) the area
integral of the divergence of a vector field $\mb{f}$ is equal to the
integral around the boundary of the field dotted with the normal to
the boundary:
%
\begin{equation}
\iint_A \nabla\vcdot\mb{f}\;dxdy = \oint_c \mb{f}\vcdot d\hat{\mb{n}}
\end{equation}

See, for example \cite{george_b._arfken_mathematical_1995},
Sect.~1.11, p 58.

This can be used to find the average value of the divergence of
$\mb{f}$ over the area of a hexagon; $\mb{f}_0$ is the average value
of the vector field at the centre of the hexagon;
$\mb{f}_1$--$\mb{f}_6$ are the average values of the field in the 6
neighbouring hexagons. The contour integral can be discretized for
this hexagonal region, which has area $\Omega$ and a side length $v$:
%
\begin{equation}
\begin{split}
\frac{1}{\Omega} \oint_c \mb{f}\vcdot d\hat{\mathbf{n}} & \approx \frac{1}{\Omega} \sum_{j=1}^{6} \frac{\mb{f}_j + \mb{f}_0}{2}\vcdot \hat{\mb{n}}\;v \\
& = \frac{1}{\Omega} \sum_{j=1}^{6} \frac{\mb{f}_j^x + \mb{f}_0^x}{2} \vcdot \hat{\mb{n}}\;v +  \sum_{j=1}^{6} \frac{\mb{f}_j^y + \mb{f}_0^y}{2} \vcdot \hat{\mb{n}}\;v \\
& = \frac{1}{\Omega} \sum_{j=1}^{6} \frac{\mb{f}_j^x + \mb{f}_0^x}{2} \cos (60\times(j-1))\;v + \frac{1}{\Omega} \sum_{j=1}^{6} \frac{\mb{f}_j^y + \mb{f}_0^y}{2} \sin (60\times(j-1))\;v \\
\end{split}
\end{equation}

An advantage to using this approach is that if it is required that the
\emph{divergence} be 0 across the boundary then for those edges of the hex
that lie on the boundary $\mb{f}_j$ can be set equal to $\mb{f}_0$
(this is the ghost cell method), satisfing this boundary condition,
whilst permitting flow through the hex parallel to the
boundary. However, it is not so easy to choose a ghost cell with
properties that will force the \emph{value} of $\mb{f}_0$ to 0.

\subsection{Computing div J: approach 2}

%
An alternative approach is to apply some vector calculus identities to
work out the divergence of $\mb{J}_i(\mb{x})$. Because the divergence
operator is distributive, Eq.~\ref{eq:divJ} can be expanded:
%
\begin{equation}
\nabla\vcdot\mb{J}_i(\mb{x}) = \nabla\vcdot\big(D \nabla
a_i(\mb{x},t)\big) - \nabla\vcdot\big(a_i \mb{g}(\mb{x})\big)
\end{equation}
%
$D$ is a constant and applying the vector calculus product rule
identity to $\nabla\vcdot\big(a_i \mb{g}(\mb{x})\big)$:
%
\begin{equation}
%
\nabla\vcdot\mb{J}_i(\mb{x}) =
D \nabla\vcdot\nabla a_i(\mb{x},t)
-
\big(
a_i(\mb{x},t)\nabla\vcdot\mb{g}(\mb{x})
+
\mb{g}(\mb{x})\vcdot\nabla a_i(\mb{x},t)
\big)
%
\end{equation}
%
(to prove $\nabla\vcdot D\nabla a = D \nabla\vcdot\nabla a$, apply the
product rule and note that $\nabla D = 0$ for constant $D$). Expanding
the brackets we have:
%
\begin{equation} \label{eq:divJExpanded}
%
\nabla\vcdot\mb{J}_i(\mb{x}) =
D \nabla\vcdot\nabla a_i(\mb{x},t)
-
a_i(\mb{x},t)\nabla\vcdot\mb{g}(\mb{x})
-
\mb{g}(\mb{x})\vcdot\nabla a_i(\mb{x},t)
%
\end{equation}
%
which has three elements to compute: the Laplacian of
$a_i(\mb{x},t)$; a time-independent modulator of
$a_i(\mb{x},t)$ [because $\nabla\vcdot\mb{g}(\mb{x})$ is a
time-independent static field]; and the dot product of the static
vector field $\mb{g}(\mb{x})$ and the gradient of
$a_i(\mb{x},t)$.

Each of the divergences can be simplified by means of Gauss's Theorem.
Reference \cite{lee_hexagonal_2014} presents an application of Gauss's
Theorem to the solution of Poisson's equation, which contains a
Laplacian term, on an hexagonal grid. The computation of the mean
value of our Laplacian, $D \nabla\vcdot\nabla a_i(\mb{x},t)$, across
the area of one hexagon located at position $\mb{p}_0$, with
neighbours at positions $\mb{p}_1$--$\mb{p}_6$ is:
%
\begin{equation}
D \nabla\vcdot\nabla a_i(\mb{p}_0,t) = D\frac{2}{3d^2} \sum_{j=1}^{6}\big(a_i(\mb{p}_j) - a_i(\mb{p}_0)\big)
\end{equation}
%
where $d$ is the centre-to-centre distance between hexes in the
grid. Let's work it through. In the following, $v$ is the length of
each of the 6 edges of the hexagon's perimeter and $v =
d/\sqrt{3}$. $d\gamma$ is an infinitessimally small distance along the
perimeter of the hexagon. The area of each hexagon in the lattice is
$\frac{\sqrt{3}}{2}d^2$.

\begin{equation}
\begin{split}
\frac{1}{\Omega} \iint_A \nabla\vcdot\nabla a_i(\mb{x}) dxdy & = \frac{1}{\Omega} \oint_c \frac{\partial a_i}{\partial \hat{\mb{n}}} d\gamma \\
& \approx \frac{1}{\Omega} \sum_{j=1}^6 \frac{\partial a_i(\mb{p}_j)}{\partial \hat{\mb{n}}} \bigg\rvert_{mid} v \\
& = \frac{2}{\sqrt{3} d^2} \sum_{j=1}^6 \frac{a_i(\mb{p}_j) - a_i(\mb{p}_0)}{d} \frac{d}{\sqrt{3}} \\
& = \frac{2}{3 d^2} \sum_{j=1}^6 \big(a_i(\mb{p}_j) - a_i(\mb{p}_0)\big)
\end{split}
\end{equation}

Applying Gauss's Theorem to the second term, the computation of
$a_i(\mb{p}_0,t)\nabla\vcdot\mb{g}(\mb{p}_0)$ can be written out in a
similar way:
%
\begin{equation}
\begin{split}
\frac{1}{\Omega} \iint_A a_i\nabla\vcdot\mb{g}\;dxdy & = \frac{a_i(\mb{p}_0)}{\Omega}  \oint_c \mb{g}\vcdot d\hat{\mathbf{n}} \\
& \approx \frac{a_i(\mb{p}_0,t)}{\Omega} \sum_{j=1}^{6} \frac{\mb{g}_j + \mb{g}_0}{2}\vcdot \hat{\mb{n}}\;v \\
& = \frac{2 a_i(\mb{p}_0,t)v}{\sqrt{3}d^2} \bigg( \sum_{j=1}^{6} \frac{\mb{g}_j^x + \mb{g}_0^x}{2} \vcdot  \hat{\mb{n}} + \frac{1}{\Omega}  \sum_{j=1}^{6} \frac{\mb{g}_j^y + \mb{g}_0^y}{2} \vcdot  \hat{\mb{n}} \bigg) \\
& = \frac{2 a_i(\mb{p}_0,t)d}{\sqrt{3}d^2\sqrt{3}} \bigg( \sum_{j=1}^{6} \frac{\mb{g}_j^x + \mb{g}_0^x}{2} \vcdot  \hat{\mb{n}} + \frac{1}{\Omega}  \sum_{j=1}^{6} \frac{\mb{g}_j^y + \mb{g}_0^y}{2} \vcdot  \hat{\mb{n}} \bigg) \\
& = \frac{a_i(\mb{p}_0,t)}{3d} \bigg( \sum_{j=1}^{6} \big(\mb{g}_j^x + \mb{g}_0^x\big) \cos (60\times(j-1)) + \sum_{j=1}^{6} \big({\mb{g}_j^y + \mb{g}_0^y}\big) \sin (60\times(j-1)) \bigg) \\
\end{split}
\end{equation}
%
The final expression is suitable for a numerical computation. Lastly,
the final term in Eq.~\ref{eq:divJExpanded} is the
scalar product of two vector fields which I'll carry out simply with
Cartesian $x$ and $y$ components - so I'll re-use my spacegrad2D
function to find $\nabla a_i$.

I think that the advantage of separating this computation out is that
I can ensure that $\mb{J}$ resulting from the first term remains 0, by
the ghost cell method, then I can tailor $\mb{g}(\mb{x})$ so that it,
and its normal derivative go to 0 at the boundary, ensuring that the
second and third terms also contribute nothing to $\mb{J}$ at the
boundary. To do this, I'll apply a fairly sharp logistic function
based on distance to the boundary to $\mb{g}(\mb{x})$.

\section {Signalling}

Gao et al. show ephrin-A5 inhibits the growth of axons in neurons of
the medial thalamus, while allowing neurons from the lateral thalamus
to branch. ``The EphA5 receptor and its ligand, ephrin-A5 are
expressed in complementary patterns in the thalamus and their cortical
targets. \emph{Additionally}, ephrin-A5 specifically inhibits neurite
outgrowth of medial thalamic neurons from limbic nuclei and sustains
neurite outgrowth of lateral thalamic neurons from primary sensory and
motor nuclei. ... Further analysis using hippocampal neurons indicated
that ephrin-A5 inhibits primarily the growth of axons detected with
antibody against the axon-specific marker $\tau$(60--70\% reduction in
axonal length), although a mild inhibition of dentritic growth also
was observed ($\approx$20\% reduction).


%
% BIBLIOGRAPHY
%
\selectlanguage{English}
\bibliographystyle{abbrvnotitle}
% The bibliography NoTremor.bib is the one exported from Zotero. It
% may be necessary to run my UTF-8 cleanup script, bbl_utf8_to_latex.sh
%%\bibliography{NoTremor}
%
% It produces this:
\begin{thebibliography}{1}

\bibitem{george_b._arfken_mathematical_1995}
{George B. Arfken} and H.~J. Weber.
\newblock {\em Mathematical {Methods} for {Physicists}}.
\newblock Academic Press, fourth edition edition, 1995.

\bibitem{karbowski_model_2004}
J.~Karbowski and G.~Ermentrout.
\newblock {\em Journal of Computational Neuroscience}, 17(3):347--363, Nov.
  2004.

\bibitem{lee_hexagonal_2014}
D.~Lee, H.~Tien, C.~Luo, et~al.
\newblock {\em International Journal of Computer Mathematics},
  91(9):1986--2009, Sept. 2014.

\end{thebibliography}

%%% Upload the *bib file along with the *tex file and PDF on
%%% submission if the bibliography is not in the main *tex file

\end{document}
